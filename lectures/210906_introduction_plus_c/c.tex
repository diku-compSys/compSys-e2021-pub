% \documentclass[handout]{beamer}
\documentclass{beamer}

\usepackage[danish]{babel}
\usepackage{amsmath,amssymb}
\usepackage{amsthm}
\usepackage{stmaryrd}
\usepackage[T1]{fontenc}

\mode<article> % only for the article version
{
  \usepackage{fullpage}
  \usepackage{hyperref}
}


\setbeamertemplate{itemize item}{\raisebox{0.8mm}{\rule{1.2mm}{1.2mm}}}
\usenavigationsymbolstemplate{} % no navigation buttons

\usepackage{tabularx}

% \usepackage{pstricks,pst-node,pst-text,pst-3d}
\usepackage{bm,latexsym,url,shadow,color}
\usepackage{graphicx}

% \usepackage{theorem}
%\include{reniers}

%%%%%%%%%%%%%%%%%%%%%%%%%%%%%%%%%%%%%%%%%%%%%%%%%%%%%%%%%%%%%%%%%%%%%%%%%%%
\usepackage{pgf}
\usepackage{tikz}
\usetikzlibrary{arrows,automata}

\usepackage{listings}
\usepackage{color}
\lstset{language=C,
                basicstyle=\ttfamily,
                keywordstyle=\color{blue}\ttfamily,
                stringstyle=\color{red}\ttfamily,
                commentstyle=\color{olive}\ttfamily,
                morecomment=[l][\color{magenta}]{\#}
}
\title{Computer Systems, B1-2 2020\\~\\Introduction to C}

\author{Troels Henriksen\\Based on slides by Michael Kirkedal Thomsen}

\begin{document}

\begin{frame}[plain]
    \titlepage
\end{frame}


\begin{frame}{Outline}
\begin{itemize}
\item Syntax - Statements/Expressions/Values
\item \texttt{main}
\item Arrays
\item Reading and writing to/from STDIO
\end{itemize}
\pause

\begin{block}{Programming is a craft}
  I will do live coding--feel free to ask questions and make comments.
\end{block}

\end{frame}

\begin{frame}{Recommended compiler flags }

\begin{block}{Compiler call}
\texttt{gcc -Wall -Wextra -pedantic -std=c11 -c [file] }
\end{block}

\begin{description}
\item[-Wall] Turns on (almost) all warnings
\item[-Wextra] Add warnings for unassigned values and more
\item[-pedantic] Makes warnings if you goes outside ISO C
\item[-std=c11] Chooses the latest C standard
\end{description}

\end{frame}



\begin{frame}[fragile]{Makefile}
Tool that makes it easy to build your program.

\begin{block}{}
\begin{lstlisting}
CC=gcc
CFLAGS=-std=c11 -Wall -Werror -Wextra -pedantic

.PHONY: clean all

cmd: updatedfile(s).c
  $(CC) $(CFLAGS) -o $@ -c $<
\end{lstlisting}
\end{block}

This will be detailed more over time.

\end{frame}

\begin{frame}[fragile]{C Style Guide -- Code is made to be read}
\begin{itemize}
Always use curly braces. The opening brace should be on the same line as the declaration.
\begin{lstlisting}
int f() {
  for (int i = 0; i < 100; i++) {
    if (i % 2 == 0) {
      // Do something.
    }
  }
  // Return something.
}
\end{lstlisting}

\item Use 2 spaces for indentation. Indent so as to make the structure of your code clear.

\item All functions must return a value. Returning void is not allowed.

\item Code must be warning-free.
\end{itemize}

\end{frame}


\begin{frame}[fragile]{The \lstinline{main} function}
\begin{itemize}
\item Simple main
\end{itemize}
\begin{lstlisting}[frame=single]
int main() {
  ...
}

\end{lstlisting}

\begin{itemize}
\item Main with arguments
\end{itemize}
\begin{lstlisting}[frame=single]
int main(int argc, char* argv[]) {
  ...
}

\end{lstlisting}

\end{frame}

\begin{frame}[fragile]{Back to reality - Return from \texttt{main} function}
\begin{lstlisting}[frame=single]
int main(int argc, char* argv[]) {
  ...
  return ???;
}
\end{lstlisting}
\pause
\begin{itemize}
\item Use the \lstinline{exit} function with returning from \lstinline{main} function
\item Lookup the codes that you need (use \texttt{echo \$?} to show result.)
\item If nothing: \lstinline{return EXIT_SUCCESS;} is assumed
\end{itemize}

\begin{lstlisting}[frame=single]
#include <stdlib.h>

int main(int argc, char* argv[]) {
  ...
  return EXIT_SUCCESS;
}
\end{lstlisting}
\end{frame}


\begin{frame}[fragile]{Statements: conditional}
\begin{lstlisting}[frame=single]
#include <stdlib.h>

int main(int argc, char* argv[]) {
  if (argc == 2) { // 1 argument
    return EXIT_SUCCESS;
  } else if (argc == 4) { // 3 arguments
    return EXIT_SUCCESS;
  } else {
    return EXIT_FAILURE;
  }
}
\end{lstlisting}
\begin{itemize}
\item This is actually a statement with a nested statement
\end{itemize}
\end{frame}

\begin{frame}[fragile]{Statements: switch-case}
\begin{lstlisting}[frame=single]
#include <stdlib.h>

int main(int argc, char* argv[]) {
  switch (argc) {
    case 2:
      return EXIT_SUCCESS;
      break;
    case 4:
      return EXIT_SUCCESS;
      break;
    default:
      return EXIT_FAILURE;
      break;
  }
}
\end{lstlisting}
\begin{itemize}
\item
\end{itemize}
\end{frame}

\begin{frame}[fragile]{Statements: switch-case alternative}
\begin{lstlisting}[frame=single]
#include <stdlib.h>

int main(int argc, char* argv[]) {
  switch (argc) {
    case 2:
    case 4:
      return EXIT_SUCCESS;
      break;
    default:
      return EXIT_FAILURE;
      break;
  }
}
\end{lstlisting}
\begin{itemize}
\item Cases only works on constant values (not expressions like conditionals).
\end{itemize}
\end{frame}


\begin{frame}[fragile]{Statements: while-loop}
\begin{lstlisting}[frame=single]
#include <stdio.h>

int main(int argc, char* argv[]) {
  int i = 1;
  while (i < argc) {
    printf("Argument number %d: %s\n",
           i, argv[i]);
    i = i + 1;
  }
}
\end{lstlisting}
\begin{itemize}
\item Use any expression
\item Easy to make it diverge
\end{itemize}
\end{frame}


\begin{frame}[fragile]{Statements: for-loop}
\begin{lstlisting}[frame=single]
#include <stdio.h>

int main(int argc, char* argv[]) {
  for (int i = 1; i < argc; i++) {
    printf("Argument number %d: %s\n",
           i, argv[i]);
  }
}
\end{lstlisting}
\begin{itemize}
\item Good for iteration
\item Keep it simple
\end{itemize}
\end{frame}


\begin{frame}[fragile]{Assignments}
\begin{lstlisting}[frame=single]
int main(int argc, char* argv[]) {
  int a = 1;
  int b = 2 + a;
  a += b; // a = a + b
  a *= 2;
  a++; // a = a + 1
  a--;
  ++a; // a = a + 1
}
\end{lstlisting}
\begin{itemize}
\item Useful short-hand writing styles.
\item Easy to understand when writing simple programs
\end{itemize}
\end{frame}


\begin{frame}[fragile]{Watch your side}
What is the values of a, b, and c?
\begin{lstlisting}[frame=single]
#include <stdio.h>

int main(int argc, char* argv[]) {
  int a = 3;
  int b = 5;
  int c = a++ + ++b;
  printf("a = %d\n", a);
  printf("b = %d\n", b);
  printf("c = %d\n", c);
}\end{lstlisting}
\pause
\begin{itemize}
\item Informally, \lstinline{++a} first increments \lstinline{a}, then returns its value
\item Informally, \lstinline{b++} first returns the value of \lstinline{b}, then increments
\item Be very careful of side-effects
\end{itemize}
\end{frame}


\begin{frame}[fragile]{Watch you side -- lets try again}
What is the values of a, b, and c?
\begin{lstlisting}[frame=single]
#include <stdio.h>

int main(int argc, char* argv[]) {
  int a = 3;
  int b = 5;
  int c = -2;
  a += (c = a++ - (b += --c)) + ++b;
  printf("a = %d\n", a);
  printf("b = %d\n", b);
  printf("c = %d\n", c);
}
\end{lstlisting}
\pause
\begin{itemize}
\item Assignments are expressions with side-effects that return the
  value that is assigned to a variable.
\item \textbf{Order of evaluation in expression is unspecified.}
\end{itemize}
\end{frame}


\begin{frame}[fragile]{Arrays}
\begin{lstlisting}[frame=single]
#include <stdio.h>

int main(int argc, char* argv[]) {
  int a[16];
  a[0] = 1;
  for (int i = 1; i < 16; i++) {
    a[i] = a[i-1] + 1;
  }
  printf("final = %d\n", a[15]);
}
\end{lstlisting}
\begin{itemize}
\item No check for out-of-bounds
\item Arrays cannot be assigned to
\item Arrays cannot be compared
\item Do it element wise
\end{itemize}
\end{frame}


% \begin{frame}[fragile]{Structs}
% \begin{lstlisting}[frame=single]

% #include <stdio.h>

% struct point {
%   int x;
%   int y;
% };

% struct point add(struct point p, struct point q) {
%   struct point w = { .x = p.x + q.x
%                    , .y = p.y + q.y};
%   return w;
% }
% ...
% \end{lstlisting}
% \begin{itemize}
% \item Create aggregate data types
% \end{itemize}
% \end{frame}


% \begin{frame}[fragile]{Structs}
% \begin{lstlisting}[frame=single]

% #include <stdio.h>

% struct point {
%   int x;
%   int y;
% };

% struct point add(struct point p, struct point q) {
%   struct point w = { .x = p.x + q.x
%                    , .y = p.y + q.y};
%   return w;
% }
% ...
% \end{lstlisting}
% \begin{itemize}
% \item Create aggregate data types
% \end{itemize}
% \end{frame}



\begin{frame}[fragile]{Writing to stdout}
\begin{lstlisting}[frame=single]
int result = fprint("Show %s of value %d ", ... );
\end{lstlisting}

\begin{tabular}{rl}
c &Character \\
d or i & Signed decimal integer \\
e & Scientific notation (mantissa/exponent) using e character \\
E & Scientific notation (mantissa/exponent) using E character \\
f & Decimal floating point \\
g & Uses the shorter of \%e or \%f \\
G & Uses the shorter of \%E or \%f \\
o & Signed octal \\
s & String of characters \\
u & Unsigned decimal integer \\
x & Unsigned hexadecimal integer \\
X & Unsigned hexadecimal integer (capital letters) \\
p & Pointer address \\
n & Nothing printed \\
\% & Character \\
\end{tabular}
\end{frame}


\begin{frame}[fragile]{Reading from stdin}
\begin{lstlisting}[frame=single]
#include <stdio.h>

int main(int argc, char* argv[]) {
  char input[20];
  printf ("Give me a string\n");
  scanf("%s", input);
  printf ("You wrote: %s\n", input);
}
\end{lstlisting}
\begin{itemize}
\item Reading value from stdin
\pause
\item Problem with writing outside buffer
\end{itemize}
\end{frame}



\begin{frame}[fragile]{Reading a string}
\begin{lstlisting}[frame=single]
#include <stdio.h>

int main(int argc, char* argv[]) {
  char input[20];
  printf ("Give me a string\n");
  fgets(input, sizeof(input), stdin);
  printf ("You wrote: %s\n", input);
}
\end{lstlisting}
\begin{itemize}
\item Use \lstinline{fgets}
\item Can limit the number of values read
\item Made for reading from files
\item stdin is just a file (somewhere)
\end{itemize}
\end{frame}




\begin{frame}[fragile]{Reading values}
\begin{lstlisting}[frame=single]
#include <stdio.h>

int main(int argc, char* argv[]) {
  int input;
  printf ("Give me a integer\n");
  scanf("%d", input);
  printf ("You wrote: %d\n", input);
}
\end{lstlisting}
\begin{itemize}
\item Reading value from stdin
\pause
\item Does not work
\end{itemize}
\end{frame}



\begin{frame}[fragile]{Reading values -- correct}
\begin{lstlisting}[frame=single]
#include <stdio.h>

int main(int argc, char* argv[]) {
  int input;
  printf ("Give me a integer\n");
  scanf("%d", &input);
  printf ("You wrote: %d\n", input);
}
\end{lstlisting}
\begin{itemize}
\item Reading value from stdin
\pause
\item Does not work
\end{itemize}
\end{frame}



\begin{frame}[fragile]{Getting values from arguments}
\begin{lstlisting}[frame=single]
#include <stdio.h>

int main(int argc, char *argv[]) {
  int input;
  if(argc == 2) {
    int res = sscanf(argv[1],"%d", &input);
    if (res == 0) {
      printf("Bad value\n");
    } else {
      printf("Input was: %d\n", input);
    }
  }
}

\end{lstlisting}

\end{frame}

\begin{frame}
\frametitle{You now know enough to get in troube}

\begin{center}
\alert{\huge{Questions?}}

\vspace{1cm}


\end{center}

\end{frame}

\end{document}


%%% Local Variables:
%%% mode: latex
%%% TeX-master: t
%%% End:
